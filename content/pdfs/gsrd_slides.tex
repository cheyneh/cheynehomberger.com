% A Very Brief Intro to LaTeX Typesetting - Graduate Student Research Day 2012
% Cheyne Homberger - cheyne.homberger@gmail.com

% ====================================================================== %
% tell latex to use the 'beamer' class for slideshows 
% the xcolor=dvipsnames lets us use more 'named' colors
\documentclass[xcolor=dvipsnames]{beamer} 


% this sets up the style of the presentation
% \usetheme{default}
\usetheme[compress]{Singapore}
\usecolortheme[named=teal]{structure}
\setbeamerfont{frametitle}{size = {\large}}


% opens up more math symbols
\usepackage{amsmath,amssymb}


% makes it easier to show source code
\usepackage{fancyvrb}

% a nicer font family than the standard
\usepackage{palatino}

% allows links within the talk, and to the internet
\usepackage{hyperref}

% custom environment - places two 'pages' side by side
\newenvironment{halfpage}{%
    \begin{minipage}[c][2cm][c]{.45\textwidth}}
    {\end{minipage}}

% saves on typing ... 
\newcommand{\vs}{\vspace{1pc}}
\newcommand{\hs}{\hspace{1pc}}

% for a macro example
\newcommand{\myint}[1]{\int_0^1 #1 \mathrm{d}x }


% the actual content starts here 
% everything before this line is called the preamble
% ====================================================================== %
\begin{document}



% textrm changes from sans serif - which is default for presentations - to serif
% (just my preference for titles)
\title{\textrm{An (Extremely Short) Introduction to \\ \LaTeX{} Typesetting}}
\author{Cheyne Homberger}
\date{\today} % will add in todays date when compiled
\institute{Graduate Student Research Day}


% note, all of this indentation is COMPLETELY UNNECESSARY. 
% I just to it to make things a bit clearer.


% 'frames' are the 'slides' of beamer
\begin{frame}
  \maketitle % makes a title slide from the title, auther, date, etc..
\end{frame}


\begin{frame}{Disclaimer}

  \begin{center}
  % only<n> makes stuff show up only on the nth part of the current slide
  \only<2>{You cannot learn \LaTeX{} in a 40 minute talk}
  \only<3>{You cannot learn \LaTeX{} in a ANY talk}
  \end{center}
\end{frame}



\begin{frame}
  \tableofcontents[hideallsubsections]
\end{frame}

\section{What is \LaTeX?}
\subsection{} % blank subsections so table of contents is less cluttered

\begin{frame}
  \tableofcontents[currentsection, hideallsubsections]
\end{frame}



\begin{frame}{Some History}
  \begin{block}{First came \TeX{}...} % blocks are 'blocks' of text with a title
    \TeX{} is a low level programming language written by Donald Knuth in 1978,
    used for specifying page layout.
  \end{block}
  \pause
  \begin{block}{...then came \LaTeX{}}
    \LaTeX{} is a collection of macros and functions written by Leslie Lamport,
    intended to make \TeX{} easier and simpler to write. 
  \end{block}
  \begin{center}
  \Large
  \uncover<3-4>{La + \TeX{}}  % only<k> means text shows up on kth layer of the frame
  \uncover<4>{ = \LaTeX{}}
  \end{center}
\end{frame}


\begin{frame}{.doc file vs text files}
  \pause
  \begin{block}{.doc files}
    When you save a file in .doc format, you save information about page layout,
    margins, fonts, text size, indentations, etc.\\
    \pause
    Even a very small amount of text, when saved, can be hundreds of kilobytes,
    and requires specific programs to view and modify. 
  \end{block}
  \pause
  \begin{block}{text files}
    A text file contains \emph{only} the text: no information about format,
    font, etc. Each letter or symbol is represented as one byte (8 bits) of data. \\
    \pause
    Text files can be viewed and modified by a huge variety of programs. 
  \end{block}
\end{frame}


\begin{frame}{.doc file vs text files}
  \begin{block}{}
    A .doc file is not a .docx file is not an .odt file is not a ...
  \end{block}
  \pause

  \begin{block}{}
    A text file is a text file is a text file
  \end{block}
  \pause
  
  \begin{block}{}
    A pdf is a pdf is a pdf
  \end{block}
\end{frame}
  

\begin{frame}{\LaTeX{} is...}
  \pause
  \begin{block}{}
    \LaTeX{} is a program which turns a text file into a pdf \\ 
    (or .dvi, or .ps, or... )
  \end{block}
\end{frame}


\begin{frame}{What is \LaTeX{}?}
  \pause
  \begin{halfpage}
  \begin{itemize}
    \uncover<2->{\item \LaTeX{} is \emph{not} a word processor}
    \vspace{1pc}
    \uncover<4->{\item Word processors require you to format your document as
    you write it (WYSIWYG)}
  \end{itemize}
  \end{halfpage}
  \begin{halfpage}
  \begin{itemize}
    \uncover<3->{\item \LaTeX{} \emph{is} a program which reads a .tex file and
    produces a pdf document}
    \vspace{1pc}
    \uncover<5->{\item \LaTeX{} separates the formatting from the content
    (WYSIWYM)}
  \end{itemize}
  \end{halfpage}
\end{frame}


\begin{frame}[fragile]{Document Structure} % fragile needed to display source
  \pause
  \verb+\documentclass{article} + \\ %verb prints text verbatim
  \pause
  \vs
  \hs Formatting options, \\
  \hs new commands,\\
  \hs additional packages to load, \\
  \hs etc.\\
  \vs
  \pause
  \verb+\begin{document}+\\
  \pause
  \vs
  \hs Content goes here.\\
  \vs
  \pause
  \verb+\end{document}+\\
  \pause
  \vs
  \hs \LaTeX{} ignores anything below this line.
\end{frame}


\section{Formatting}
\subsection{}

\begin{frame}
  \tableofcontents[currentsection, hideallsubsections]
\end{frame}

\begin{frame}[fragile]{Text Size and Style}
\pause
  \begin{block}{Text groups}
    Text inside brackets \verb+ { ... } + is called a \emph{group}.\\
    Style commands are typically confined to a specific group. 
  \end{block}
  \pause

  \begin{block}{Text Options}
    \url{http://en.wikibooks.org/wiki/LaTeX/Text_Formatting#Font_Styles}
  \end{block}

  \pause
  \begin{block}{Emphasis}
    \verb+\emph{ ... }+ will automatically emphasize text differently depending
    on the context. 
  \end{block}
\end{frame}


\begin{frame}[fragile]{More Text Stuff}
  \pause
  
  \begin{block}{Aligment}
    \verb+\begin{center} ... \end{center}+ centers text. Replace `center' with
    `flushright' or `flushleft' to justify.
  \end{block}

  \pause

  \begin{block}{Extra space}
    \verb+\vspace{1in}+ gives one inch of vertical space, \verb+\hspace{1in}+
    gives horizontal space. \verb+\vfill+ and \verb+\hfill+ fill up vertical and
    horizontal space on the page. \\
    \url{http://en.wikibooks.org/wiki/LaTeX/Useful_Measurement_Macros}
  \end{block}

  \pause

  \begin{block}{Starting a new page}
    \verb+\pagebreak+ \emph{suggests} to \LaTeX{} to start a new page here.\\
    \pause
    \verb+\pagebreak[4]+ \emph{strongly suggests} to start a new page. \\
    \pause
    \verb+\newpage+ forces a new page. 
  \end{block}
\end{frame}


\begin{frame}[fragile]{Bullet Points}
\pause
  \hs \hs
  \begin{halfpage}
\begin{Verbatim}[frame=single, fontsize=\small]
\begin{itemize}
  \item first
  \item second
  \item third
\end{itemize}

\begin{enumerate}[i)]
  \item first
  \item second
  \item third
\end{enumerate}
\end{Verbatim}
  \end{halfpage}
  \pause
  \begin{halfpage}
  \begin{itemize}
    \item first
    \item second
    \item third
  \end{itemize}

  \begin{enumerate}[ i) ]
    \item first
    \item second
    \item third
  \end{enumerate}
  \end{halfpage}
\end{frame}







\section{References}
\subsection{}

\begin{frame}
  \tableofcontents[currentsection, hideallsubsections]
\end{frame}


\begin{frame}[fragile]{Internal References}
  \pause
  \begin{block}{Labels}
    Defintions, bullet points, figures, etc. can be \emph{labelled}. Labels are
    only visible in the source code, and will stay in place even when item
    numbers change. 
  \end{block}
  \pause
  \begin{Verbatim}
  \begin{figure} \label{maingraph}
  ...
  \end{figure}
  \end{Verbatim}
  \pause
  \begin{Verbatim}
  ...
  ...
  As we have seen in Figure \ref{maingraph} 
  on page \pageref{maingraph}, ...
  \end{Verbatim}
  \pause
  \verb+\usepackage{hyperref}+ makes each reference \emph{clickable}
\end{frame}


\begin{frame}[fragile]{Table of Contents and Index}
  \pause
  \begin{block}{Sections and Subsections}
    If you use \verb+\section{...}+, \verb+\subsection{...}+\\
    \verb+\chapter{...}+, etc in your document, then \\
    \verb+\tableofcontents+ generates a full table of contents. 
  \end{block}
  \pause
  \begin{block}{Index}
    With package \verb+makeidx+, use \verb+\index{key}+ next to topics in your
    paper. The command \verb+\makeindex+ will generate a full index. 
  \end{block}
\end{frame}

\begin{frame}[fragile]{Bibliographies}
  Bibliographies in \LaTeX{} work very similarly. Each item is given a name, and
  cited with \verb+\cite{name}+. \\
  \pause
\begin{Verbatim}[fontsize=\small]
\begin{thebibliography}
  \bibitem{name}
    author, title, journal, etc.
\end{thebibliography} 
\end{Verbatim}

  \pause
  bibTeX is a separate (but closely related) program which completely manages
  your bibliography. It consists of a separate file (with a .bib) extension. \\
  \pause
  bibTeX will \emph{automatically} style your references, order them
  alphabetically, and only list those you actually cite (or list them all, if
  you want). 
\end{frame}


\begin{frame}[fragile]{BibTeX}
\begin{Verbatim}[frame=single]
@article{infprimes,
    AUTHOR = {Euclid},
     TITLE = {On the Infinitude of Primes},
   JOURNAL = {First Journal of Math},
    VOLUME = {1},
      YEAR = {300 BC},
     PAGES = {1 - 5},
} 
\end{Verbatim}

  \pause
  \verb+\cite{infprimes}+ will now generate this reference, number it, and cite
  it. 
\end{frame}
 

\section{Advanced Topics}
\subsection{}

\begin{frame}
  \tableofcontents[currentsection, hideallsubsections]
\end{frame}

\begin{frame}[fragile]{Macros and Shortcuts}
  \pause
  \begin{block}{New Commands}
    \verb+\newcommand{\me}{Cheyne}+ tells \LaTeX{} to replace \verb+\me+ with
    \verb+Cheyne+. 
  \end{block}
  \pause
  \begin{block}{New Commands}
    $$\int_0^1f(x) \mathrm{d}x $$
    \pause
  \verb+\newcommand{\myint}[1]{\int_0^1 #1\mathrm{d}x}+ \\
    \vs 
    \verb+\myint{ e^x }+ gives $\myint{e^x}$. 
  \end{block}
\end{frame}


\begin{frame}[fragile]{Presentations}
  
  \pause
  \emph{Beamer} is a document class for producing presentations and posters
  \pause

\begin{Verbatim}[fontsize=\small]
\documentclass{beamer}
\usetheme{...}
\usecolortheme{...}

\begin{document}

\begin{frame}{frametitle}
  stuff
  \pause
  more stuff 
\end{Verbatim}
\vspace{-.6pc}
{\small \verb+\end{frame}+}
\end{frame}



\end{document}


